%!TEX root = spadini_davide.tex

\chapter*{Introduction} % senza numerazione
\label{sommario}

\addcontentsline{toc}{chapter}{Introduction} % da aggiungere comunque all'indice
Gossip-based communication protocols have been successfully applied in large scale systems with important applications which include information dissemination, aggregation, overlay topology management and synchronization. The common property of these protocols is that each node in the system periodically exchanges information with a subset of its peers. The choice of this subset is crucial to the wide dissemination of the gossip. Ideally, any given node should exchange information with peers that are selected following a uniform random sample of \textit{all} nodes currently in the system. However, providing each node with a complete membership table from which a random sample can be drawn, is unrealistic in a large-scale dynamic system, because maintaining such tables in the presence of churn (meaning that nodes can join or leave at any given point in time) incurs considerable synchronization costs. 

So the Peer Sampling Service (PSS), the underlying service that provides each node with a list of peers, is a fundamental distributed component of gossip-based protocols. A PSS can be implemented as a centralized service, using gossip protocols or random walks. Gossip-based PSSes~\cite{gossip_protocol} have been the most widely adopted solution, as centralized PSSes are expensive to run reliably, and random walks are only suitable for stable networks, i.e. with very low levels of churn. 

In order to handle situations in which the nodes can not communicate directly with one another, for example because the nodes reside behind Network Address Translation Gateways (NATs) and firewalls, there are NAT-aware gossip-based PSSes that are able to generate uniformly random samples even for systems with a high percentage of private nodes, that is, nodes that reside behind a NAT and/or firewall. 

State of the art of NAT-aware gossip protocols, such as Gozar~\cite{gozar}, require peers to frequently establish network connections and exchange messages with public nodes, nodes that support direct connectivity, in order to build a view of the overlay network. This design is based on two assumptions: the first one is that the connection establishment from a private to a public peer comes at negligible cost, and the second one is that the connection setup time is short. However, in many Peer-To-Peer applications like for example Google WebRTC~\cite{webrtc}, these assumptions do not hold. In general, establishing a connection is a relatively complex and costly procedure: for security reasons but also to overcome the problems of NATs traversal. 

In 2013 has been presented the Wormhole Peer Sampling Service~\cite{wormhole}, which provides the same properties of other PSSes (freshness of samples, randomness, robustness to different level of churn, NAT-friendliness) while it decreases the connection establishment rate by one order of magnitude. 

The goal of this thesis is to provide an alternative implementation of this algorithm, using WebRTC as ``network handler''. WebRTC is an Application Programming Interface (API) definition that supports browser-to-browser applications. It is already implemented in the Chrome, Firefox and Opera browsers, and it is perfectly suitable for our purpose because it could be used from multiple types of device (personal computer, but also smart-phones) since it works on browsers. In this way our implementation could be used to build a peer-to-peer application connecting smart-phones to a personal computer.

We will show that our implementation maintains the same properties as the original while it is more robust to churn. We also give an implementation of a gossip-based aggregation protocol that is built on top of our PSS, testing it with a lot of experiments and measuring the convergence time.

In Section~\ref{cha:webrtc} we give an introduction on WebRTC, in Section~\ref{cha:wormhole} we explain the WPSS algorithm, in Section~\ref{cha:implementation} we show the main parts of our implementation and the differences from the original one, in Section~\ref{cha:evaluation} we evaluate our protocol through a lot of tests in a simulation environment, we also include in Section~\ref{cha:aggregation} the implementation of a gossip-based aggregation protocol with some evaluation tests. In the final conclusions, Section~\ref{cha:conclusions}, we discuss the obtained results, the limitations of our model and some suggestions for possible solutions.
