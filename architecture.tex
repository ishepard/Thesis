\chapter{Project Architecture}
\label{cha:arc}
We implemented and tested the project in a real environment and also in simulation. WebRTC are web browser APIs, meaning that they can only work in a browser session. So we developed a first version of the WPSS that works in the browser, and tested it with few nodes (100) using Google Chrome\footnote{Google Chrome is a free-ware web browser developed by Google} as browser. Then, in order to test it in a bigger environment we use a simulator. Both the versions are implemented using the framework \textbf{Hivejs-Framework} provided by the creator of the algorithm, the Hive Streaming\footnote{Hive Streaming provides network solutions for media distribution and performance analysis. Started as a spin-off in 2007 from the Swedish Institute for Computer Science and the Royal Institute of Technology in Stockholm, the company maintains a strong focus on research and development. \url{https://www.hivestreaming.com}} company. This framework is written in Typescript\footnote{TypeScript is a typed superset of JavaScript that compiles to plain JavaScript. \url{http://www.typescriptlang.org}}, and it enables us to choose if we want to run the project on browser or directly in \textit{Node.js} simulating a lot of instances. 

The Hivejs-Framework has a WebRTC module for handling peer-to-peer connections which, as explained in Sect.\ref{sec:easy_tc}, is EasyRTC. 